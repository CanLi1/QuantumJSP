\documentclass[letterpaper, 11pt]{article} 

\usepackage{graphics,graphicx}
\usepackage{multicol} 
\usepackage{parskip}
\usepackage{amsmath}
\usepackage{multirow}
\usepackage[english, polish]{babel}
\usepackage[utf8]{inputenc}
\usepackage{fancyhdr}
\usepackage[title]{appendix}
\usepackage{wasysym}
\usepackage{url}

\usepackage[font=footnotesize,labelfont=small]{caption}
\captionsetup{width=0.85\linewidth}

\RequirePackage{geometry}
\geometry{margin=2cm}

\selectlanguage{english}
\setlength{\parskip}{0.2cm}
\setlength{\parindent}{0pt}


%----------------------------------------------------------------------------------------
%	BASIC INFO
%----------------------------------------------------------------------------------------


\title{Report of our work on Job Shop Scheduling using D-Wave System's Quantum Annealing}
\author{
Marek Subocz, Krzysztof Kurowski \\
\textit{Poznan Supercomputing and Networking Center (PSNC)}
}
\date{10 Oct 2019}
%--------------------------------------------------------------------------------CUERPO-----------------------------------------%
\begin{document}
\maketitle
We attempted to harness Quantum Computing speed to improve the 
performance of already existing heuristics, particularily 
for Job Shop Scheduling Problem. Our main concern was to bypass
its small number of computing units, and therefore small size
of a computable instance, to acquire the advantage of quantum
mechanics in D-Wave's quantum annealing algorithm. 
\vspace{10pt}
\begin{multicols}{2}
\section{Introduction}
\label{sec:intro}
\subsection{Problem definition}
The JSP is an minimization problem consisting of a set of jobs 
$J = \{\textbf{j}_1,\dots,\textbf{j}_N\}$
that must be scheduled on a set of machines
$M = \{\textbf{m}_1, \dots, \textbf{m}_M\}$. 
Each job consists of a sequence of operations that have to be performed
in a predefined order:
\begin{equation}
\textbf{j}_n = \{O_{n1} \rightarrow O_{n2} \rightarrow \dots \rightarrow O_{nLn}\}
\end{equation}
Each operation has only one machine it can be performed on and an
integer execution time $t_{nj} \geq 0$.
Also, there can only be one operation running on a given machine at 
any given point in time and each operation of a job needs to complete 
before the following one can start. The main objective is to schedule
all operations in a valid sequence while minimizing total schedule time.
\\
In our algorithm we use a repeatedly executed sequence to hopefully
fully explore a small instance's possibilities. To do this, we convert
a given JSP problem to Quadratic Unconstrained Binary Optimization
Problem (QUBO). 

\subsection{QUBO Problem Formulation}
With our approach we define a given JSP instance with $n_O * T$
binary variables, where $n_O$ is the total number of operations and
$T$ is the upper bound for considered scheduling times. Every
operation has a binary variable for every point in time:
\begin{equation}
    x_{i,t} =
    \begin{cases}
        \text{ 1 : operation $O_i$ starts at time $t$,} \\
        \text{ 0 : otherwise.}
    \end{cases}
\end{equation}
\subsection{Constraints}
It would be too expensive and imprecise to perform a quantum 
annealing computation on all these variables and combinations of
possible outcomes. Because of that, we put some constraints on
what is regarded as a viable solution and eliminate (lock on the
value of 0) some of the variables completely.
\subsubsection{One start constraint}
Every operation needs to have just one start point and therefore all of
it's variables need to sum to one. 
\subsubsection{Precedence constraint}
Every operation needs to start after the previous one from the same job
has ended. We make sure of that by checking if every operation's 
start point occurs at $previous\_start\_point + previous\_length$ or later.
\subsubsection{Share machine constraint}
Every machine can perform up to one operation at a time. To assure that
this constraint is not violated, we group all of the variables by it's
corresponding machines and check if the operations interfere in any way.
\subsection{Variable pruning}
Some of the QUBO variables can be eliminated in a precomputing phase,
improving the robustness of quantum annealing. 
% TODO: graph edge cutout picture
\subsubsection{Too early for a task}
We disable all of the variables $0 \leq x_{i,j} < S$, where $S$ is the sum of lengths of all the operations \textbf{prior to} the considered one in the same job.
\subsubsection{Too late for a task}
We disable all of the variables $T-S \leq x_{i,j} < T$, where $S$ is the sum of lengths of all the operations \textbf{after} the considered one in the same job.
\subsubsection{Disabled regions}

\subsubsection{Other disabled variables}

% TODO: liczba i topologia połączeń kubitów

\subsection{Greedy algorithm}
In order to use our heuristic approach, we need to already have a solution
that we will improve. While the kind of such a solution that is the best for
combining a substantial diversity of possible solutions and starting reliably
close to an optimum is a subject for debate,

\end{multicols}
\end{document}